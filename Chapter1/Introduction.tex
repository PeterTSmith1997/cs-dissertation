
The use of real time attack provention software is widely accepted, the is a lot of softwre able to do this. It is easy to idenify and stop realtime large scale attacks for example a DDOS. However thease are normally used by large scale hosting providers and other services for example Cloudflare(ref here). Whilst thease are normaly very effective at blocking such attacks, there are two flaws, the main being that thease systems only relay on realtine data and traffic and the data is not provided to end users.

This work will look at issues around whether non cyber security experts can identify attacks and how this should be built into a user friendly piece of software to identity low-rate bandwidth attacks.

This work will aim to propose a formula that can accurately identify a variety of attacks however, the formula will need to be generic enough to be used on a variety of websites. Due to the varying nature of low-rate bandwidth attacks, this work will look at current methodologies for detecting these types of attacks and critically review them.
The Conclusions from the 2016 study by Erwin Adi suggested that the protocol itself does not restrict the intensity of traffic generated. The high-lit the need for further research and the potential application of a further protocol or mechanism with the aim of identifying the volumes and patterns of network traffic communicated between a client and network machine during simulated DoS attacks. After consideration of the conclusions from the Erwin Adi studies I decided to investigate the potential of creating an early detection.
