\section{Data not included in the Formula}

The formula as previously shown is able to detect attacks with a fairly high degree of accuracy however, during the testing it became apparent that some data was not taken into account. The protocol used for the request; this could have been added to indicate whether the connection was secure due to the fact the HTTP2 only works on a secure connection therefore, this could have been used as an indicator of how secure the request was and maybe increase the risk if HTTP1 was used and detracted a HTTP2 was used. It could however be argued that more risk factor could have been given to HTTP2 due to the fact that \cite{tripathi2018slow} pointed out that HTTP2 had more threat vectors; due to this uncertainty it was decided it was better to leave the protocol out of the formula. 

Due to the time constraints on this project, the time of day of the traffic was not taken into account; this was because to accurately use this, the local time of the IP address would have needed to be calculated and then a further risk factor plan is required. however, due to the 24 hour nature of the internet this would be difficult to apply risk to.

.We did not look at HTTP1/HTTP2 
.We did not look at the time of day of the requests
.we did not look at 301 error 302, 402
.