\section{Data not included in the Formula}

The formula as it stands is able to detect attacks with a fairly high degree of accuracy, however, during the testing it became apparent that some data was not taken into account. It should be mentioned that for the purposes of this formula, the difference in HTTP1 and HTTP2 in terms of security has not been assessed and appraised with a relevant risk rating. This risk related element could have been added to indicate whether the connection was secure mainly due to the fact that HTTP2 only works on a secure connection. This could have been used as an indicator of how guarded the request was, HTTP1 can operate on an unsecured connection, therefore the data that is transmitted is unencrypted and less secure. It could be argued that a greater risk factor could have been given to HTTP2, \cite{tripathi2018slow} pointed out that HTTP2 had more threat vectors than HTTP1. Due to the evolution and the work in progress state of HTTP2, it was decided that it was a better idea to leave the risk related appraisal of these protocols out of the formula. This was mainly due to the fact that there was no way to guarantee that the HTTP2 protocol remained unpatched and unaltered throughout the timeline of this research.

Due to the time constraints of this project, certain factors that should have been considered were left out. Firstly, the timezone from which the log files originated, and secondly the timezone from which the IP hailed. This would allow the software to calculate the correct time of day that traffic appeared on the website. This was largely down to the fact that to accurately use this data, the local time of the IP address would have needed to be calculated, then a further risk factor plan would be required. This would be a difficult factor to apply risk to, due to the 24 hour nature of the internet.

It would be worthwhile mentioning that certain return codes were not allocated risk ratings as part of the request/response factor within the formula. These in general are not considered to be risk related returns and are mainly indicative of page redirects and paid content response, hence, the omission of these factors from the formula was considered of negligible effect.
