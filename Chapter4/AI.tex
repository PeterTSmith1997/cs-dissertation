\section{Introduction}
In previous chapters it has been shown, how attacks can be identified and mitigated using algorithms and other machine learning in particular, high rate bandwidth attacks by using systems such as cloudflare. The majority of research approaching the subject of the detection or mitigation of malicious traffic have focused upon programmed methods of automated detection. To date there is a distinct lack of data on the human element for detection methodology, two main areas will be assessed in the role of the human in the system. This chapter shall look into the dynamics of human, computer interaction with a view of identifying key issues in threat detection methodology. Any lessons that can be learned from the literature regarding good practice for user interface design shall be acknowledged and applied in the design of a better purposed user interface for the software. 

\section{Relationship between human and AI}

Crannor (2008) stated that 'overall humans are a major cause of computer security failures' and proposed that 'Automated components are generally more accurate and predictable than humans.'(\cite{cranor2008framework}) This indicates that Crannor is opposed to keeping a "human in the loop" when designing security systems. Crannor also contradicts himself by suggesting that a human being may be a better judge of "context" than a computer, for example, whether an email attachment is suspicious in a particular context. This is supported in a whitepaper by F-Secure Global which suggests that context is everything – in life and in cyber security. Therefore, a perfect combination for cyber security well-being is formed with a partnership between man and machine (\cite{TargetedCyberSecurity}). The white paper indicates that a complete AI system would always be inferior in its detection ability over a well designed security system amalgamating human and computer symbiosis. 

Furthermore, Crannor in 2008 seems to disagree with non expert users being able to identify attacks without sufficient support and or guidance. Instead Crannor suggests that human involvement may make the system as a whole, less secure. The paper, identifies 3 categories of moderator profiles, who may potentially have a negative effect on system security processes as a whole. The controllers in question were stated to be; 'Non-malicious humans who don't understand when or how to perform security-related tasks, humans who are unmotivated to perform security-related tasks or comply with security policies, and humans who are  not capable of making sound security decisions.' (\citeauthor{cranor2008framework} \citeyear{cranor2008framework}) This is in contrast to the work carried out by \citeauthor{ben2015effects} which would suggest that non experts were able to identify malicious activity to a degree of accuracy, although falling short of the reliability of that of a Cyber security expert (\cite{ben2015effects}).
 
Crannor's prior work in 2005 when researching the subject of Security and usability did suggest that 'high usability' and 'high security' are not mutually exclusive design features. Crannor pointed out that usability and security are often misconceived to be competing features when designing security software (\cite{cranor2005security}).

Overall, Crannor proposed three areas to improve existing security systems. The first suggested the removal of the human element from the loop completely, in which case applying a full reliance on the software of an AI. Crannor's second suggestion for improvement was to implement systems that are intuitive, and simple to use in order to support a novice to engage with them. Crannor's third suggestion was to implement training to novice users in order to educate them on the principals of cyber security. It should be noted however that Crannor did suggest that some novice users may not be 'all that receptive to learning' (\cite{cranor2008framework}). These are important points that shall be consulted during the design of the user interface in a latter part of this paper (section \ref{ui}), and a balance will need to be reached. 