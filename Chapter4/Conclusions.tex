\section{Conclusions} 
 
 After considering the research into the fields of user understanding and human-computer interaction many key principals were uncovered. Human behaviour in general is particularly complicated to understand or predict. This will present many issues during the design of functionality for the user interface. As discussed in many of the papers the base knowledge of the users in particular makes a huge difference on their ability to utilise software in general. It is the general consensus that a clear and concise user interface will benefit novice users, and shall improve the overall success in identifying malicious traffic as a whole. Although many experts, such as \citeauthor{cranor2008framework}, suggested that the creation of a completely automated AI system would lead to a diagnostic system that was less likely to miss incoming threats. Some other experts, namely \citeauthor{TargetedCyberSecurity}, pointed out that the unique 'human' ability to appraise the contextual features of a potential threat means that removing them from the loop of a security methodology is inadvisable. This is mainly due to the current level of sophistication in AI technology to date. A key area that will need to be considered is the amount of human involvement when designing the software. It should also be noted that when considering the implementation of a suitable UI, we appraise the potential that novice users may be accessing this.
 
 