\subsection{Attack Detection Techniques}

Due to the similarities that are printed in the previous section, port scanning should be relatively easy to detect in most cases. Unlike low bandwidth attacks, there is a long history of detailed detection methods and the evolution of these techniques will be discussed...

An early port scanning detection technique was Snort. It was created as an open source network intrusion detection based on Libpcap. The detector used in Sanford's work attempts to look for (x) TCP or UDP packets sent to any number of ports from a single source in (Y) seconds and assesses the behaviours of this activity for notable malicious scanning characteristics. The Snort program also looks for single TCP packets that are not used in genuine TCP operations which may have unusual TCP flag settings or no flags set at all. There are several shortfalls of this program. Firstly it is unable to detect scans coming from multiple hosts. The second weakness is that the threshold is determined  and evaluated by using a static combination of user specific data (\cite{staniford2002practical}).This method is easy to deceive due to the fact that the Y value has to be set high enough so that there are very few false positives. Therefore, incoming attacks can avoid detection relatively easily, if the attacker increases the time between each successive packet.

A study conducted by \citeauthor{laroche2009evolving} in \citeyear{laroche2009evolving}, attempted to create a stealthy scanning approach to test the limitations of the Snort detection program. A mimicry model was created using a type of computer learning defined as Genetic Programming (GP). Stealthy portscan models were generated and launched with valid attributes (i.e valid TCP/IP packets). The researchers demonstrated that they successfully applied Genetic programming and produced a stealthy mechanism of port scanning that was able to fool the Snort system (\cite{laroche2009evolving}). This research shows that snort was not a reliable way of detecting portscan attacks due to the evolving techniques that can be implemented by attackers, however, Staniford in his 2002 paper proposed a second technique called Spice.
 
Spice is made up of two parts, an anomaly sensor and a correlator component. The researchers noted that saving information about packets would have a detrimental effect on performance, hence they took the approach of storing the details of more likely candidates for malicious activity for longer. The anomaly sensor looked as the suspicious features of the following; header fields, source IP, destination IP, source port, destination port, protocol and protocol flags. The researchers used a Bayesian Network statistical approach to assess the probability of anomalous entries in order to score events in their likelihood of being malicious port scan attacks. Finally, a heuristic analysis is carried out as part of the correlation process where the data is evaluated. It was noted by the research team that a limitation of the Spice detection method is that port scanning attacks that are similar in nature to normal traffic could bypass the anomaly detector. Hence, resulting in a stealthy scan attack not being identified as a threat(\cite{staniford2002practical}); this in turn makes Spice a fundamentally flawed detection method due to the evolving nature of port scan attacks, and the advanced mimicry implementation discussed in the prior section.

\citeauthor{kim2008slow} defined stealth scans as a technique that avoided IDS/IPS and logging. They also suggests that stealth methodology has evolved into the setting of flags and the interval between scans in order to appear as normal traffic (.\cite{kim2008slow}). The researchers went on to propose a novel detection method using fuzzy logic in order to categorise incoming port traffic and categorise the architecture of the traffic into 3 distinct classifications. It should be noted that in their approaches the authors suggested some drawbacks with this methodology appreciating that there is no standard of judgement about slow port scan attacks, and there is always a degree of uncertainty while using this technique to detect slow port scans.

(needs finalising)