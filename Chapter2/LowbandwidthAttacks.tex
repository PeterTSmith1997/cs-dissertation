%!TeX root=Dissertation.tex

\section{Low-bandwidth attacks} \label{attack1}

In order to comprehend the nature of the topics, and the difficulty in detecting Low-bandwidth attacks some other fundamental concepts should be illustrated within the body of this introduction for supplemental understanding of many of the research methods discussed.

Adi's 2017 paper uses a SYN flag variable as a benchmark for threat detection. To understand this variable it we must discribe the mechanics of a TCP connection. TCP is a connection-oriented protocol, a connection needs to be established before two devices can communicate. TCP uses a process called three-way handshake to negotiate the sequence and acknowledgment fields and start the session. Two devices would be in communication, the Client and the Server. The Client initiates the connection by sending the TCP SYN (SYNchronize) packet to the destination Server and the Server receives the packet and responds with an acknowledgement. The Client then acknowledges the response of the Server by sending the acknowledgment back, and a connection is formed. 

A low-rate and low bandwidth attack is a different kind of attack in comparison to a normal DoS attack. A DoS attack works on overwhelming the server with requests, therefore it is more difficult to detect than a high-rate attack. For example, if you receive a large amount of traffic from a single IP; this is more than likely a DoS attack (REF NEEDED). A normal internet request operates based on an HTTP GET request to the web server that allows access to the site. The outcome of a DoS attack is the creation of a disruption between the clients and the web host. High-bandwidth attacks will keep trying to request multiple web pages at the same time to overwhelm the server however, a low-bandwidth attack sends a partial request and then may wait 20-30 seconds for example, it then sends new data, just enough to keep the connection open. One type of a Low-bandwidth attack is a slow loris attack; this is a layer-7 application attack, that only requires a small level of band-width and thus means the attacker can have continual use over their system and carry out normal activity. A web server will have a set number of sockets that it can have open at any one time, for this explanation we will choose to use 10 sockets. The aim of a slow loris is to open all 10 sockets and keep them busy, therefore, no new sockets can be opened and thus, no client interaction will take place. The difficulty with detecting these types of attacks is that the legitimate user may just have a slow internet connection, thus, making distinguishing these attacks from slow users difficult. 

%SYN CRap Here

%\subsection{HTTP/1.1 versus HHTP/2}

%\textcolor{blue}{This study will look at previous works assessing the strengths and weaknesses of HTTP/1.1 and HTTP/2 in terms of their ability to function during a low rate DoS attack. The HTTP/2 protocol is defined as RCF 7301, therefore any further mentions of HTTP/2 will be in accordance with RCF 7301, similarly, when talking about HTTP/1.1 it will be in accordance with RCF 2616.}


%\textcolor{blue}{Although HTTP/2 is a more modern protocol the consensus is that it is a more vulnerable protocol when it comes to low-rate DoS attacks, as it has more threat vectors (\cite{tripathi2018slow}). Adi also notes that "The HTTP/2-standard states that if the host machine does not monitor resource usage, it exposes itself to a risk of a DoS attack". (\cite{Adi2015}).}