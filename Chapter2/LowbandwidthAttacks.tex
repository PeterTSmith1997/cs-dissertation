%!TeX root=Dissertation.tex

\section{Low-bandwidth attacks} \label{attack1}

A low and slow attack is a type of DoS or DDoS attack that relies on a small stream of very slow traffic which can target application or server resources. A DoS attack is a denial of service attack where a computer (or computers) is used to flood a server with TCP and UDP packets. A DDoS attack is where multiple systems target a single system with a DoS attack. The targeted network is then bombarded with packets from multiple locations. Unlike more traditional brute-force attacks, low and slow attacks require very little bandwidth and can be hard to mitigate, as they generate traffic that is very difficult to distinguish from general traffic. Because they don’t require a lot of resources to pull off, low and slow attacks can be successfully launched using a single computer. In comparison, a high rate attack is launched from numerous compromised devices, often distributed globally in what is referred to as a botnet. A botnet refers to a group of computers which have been infected by malware and have come under the control of a malicious actor. It is distinct from low rate attacks, in that it uses a single Internet-connected device (one network connection) to flood a target with malicious traffic. 


Low rate and low bandwidth attacks differ greatly from a High rate DoS attack (High bandwidth attacks shall be discussed in depth in Chapter 3). A DoS attack, works on overwhelming the server with requests, therefore, it is easier to detect than a low-rate attack. An example of this would be if you were to receive a large amount of traffic from a single IP; this more than likely would be a DoS attack. A normal internet request operates based on an HTTP GET request to the web server that allows access to the site, the outcome of a DoS attack is the creation of a disruption between the clients and the web host. 

High-bandwidth attacks will keep trying to request multiple web pages at the same time to overwhelm the server. In comparison, a low-bandwidth attack  will send a partial request and may wait 20-30 seconds, it then sends new data, just enough to keep the connection open. One type of a Low-bandwidth attack is a slow loris attack; this is a layer-7 application attack, that only requires a small level of band-width and thus means the attacker can have continual use over their system and carry out normal activity. A web server will have a set number of sockets that it can have open at any one time, for this explanation we will choose to use 10 sockets. The aim of a slow loris is to open all 10 sockets and keep them busy, therefore, no new sockets can be opened and thus, no client interaction will take place. The difficulty with detecting these types of attacks is that the legitimate user may just have a slow internet connection, thus making it difficult to distinguish these attacks from slow users. 
