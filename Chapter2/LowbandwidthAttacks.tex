%!TeX root=Dissertation.tex

\section{Low-bandwidth attacks} \label{attack1}

In order to comprehend the nature of the topic, and the difficulty in detecting Low-bandwidth attacks some other fundamental concepts should be illustrated within the body of this introduction for supplemental understanding of many of the research methods discussed.

A paper written by Adi 2017 uses a SYN flag variable as a benchmark for threat detection; to understand this variable the mechanics of a TCP connection must be described. TCP is a connection-oriented protocol, a connection needs to be established before two devices can communicate. TCP uses a process called three-way handshake to negotiate the sequence and acknowledgment fields and start the session. Two devices would be in communication, the Client and the Server. The Client initiates the connection by sending the TCP SYN (SYNchronize) packet to the destination Server and the Server receives the packet and responds with an acknowledgement. The Client then acknowledges the response of the Server by sending the acknowledgment back, and a connection is formed. 

A low-rate and low bandwidth attack are different kinds of attacks in comparison to a regular DoS attack. A DoS attack works on overwhelming the server with requests, therefore it is easier to detect than a low-rate attack. An example of this would be if you were to receive a large amount of traffic from a single IP; this would then more than likely be a DoS attack. A normal internet request operates based on an HTTP GET request to the web server that allows access to the site, the outcome of a DoS attack is the creation of a disruption between the clients and the web host. High-bandwidth attacks will keep trying to request multiple web pages at the same time to overwhelm the server. In comparison, however, a low-bandwidth attack  will send a partial request and may wait 20-30 seconds, it then sends new data, just enough to keep the connection open. One type of a Low-bandwidth attack is a slow loris attack; this is a layer-7 application attack, that only requires a small level of band-width and thus means the attacker can have continual use over their system and carry out normal activity. A web server will have a set number of sockets that it can have open at any one time, for this explanation we will choose to use 10 sockets. The aim of a slow loris is to open all 10 sockets and keep them busy, therefore, no new sockets can be opened and thus, no client interaction will take place. The difficulty with detecting these types of attacks is that the legitimate user may just have a slow internet connection thus, making distinguishing these attacks from slow users difficult. 
