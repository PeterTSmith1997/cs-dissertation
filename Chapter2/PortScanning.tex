%!TeX root=Dissertation.tex
\section{port scanning}

\textbf{A port scanner attack is an attack whereby the attacker will send packages of information to different ports on a server to see which ports can be accessed. These types of probe attacks are not expected to be impactful on their own, however they can be a prelude to a further more serious attack such as a brute force attack. It is important to note that the purpose of ports is varied and some ports expect a much higher volume of traffic than others. For example: Port 80 (unencrypted) and 443 (encrypted), deal with website traffic and are generally expected to be heavily used. A popular scan target for attackers is the SSH22 port which is essentially a gateway to an entire network. An open SSH port can result in a probe attack being followed up with a hostile attempt to take control of the network. A lot of web hosts and sever owners change the ports for SSH and the ftp ports, however, they may inadvertently leave ports open that are not secure. It is important to note that all port scanning attacks utilize the TCP 'three way handshake' that was explained in the prior section. (\ref{attack1})}

\textbf{Overview hereeeeeee}

%https://ieeexplore.ieee.org/stamp/stamp.jsp?tp=&arnumber=6122824