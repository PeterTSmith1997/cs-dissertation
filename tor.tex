%!TeX root=TermsOfReference.tex

\section{Background}

\section{Proposed Work}
\label{proposed}
\section{Aims and Objectives}
There should only be one or two aims
\subsection{Aims}
\begin{quote}
	To show how \LaTeX\ and tools can be used to write a dissertation
\end{quote}

\subsection{Objectives}
Your objective list is a series of measurable objectives, can you tick each one off as \emph{done}?  I usually expect between 8 and 12 objectives

\begin{enumerate}
    \item 
\end{enumerate}

\section{Skills}
\begin{enumerate}
\end{enumerate}

\section{Resources}

\subsection{Hardware}

\subsection{Software}

\section{Structure and Contents of the Report}
\subsection{Report Structure}

\paragraph{Introduction}  Sets out the background and motivation for the project.  Summarises the work done, the results, the conclusions, and the recommendations for future work.  It is a one chapter summary of the \emph{entire} project.

\paragraph{Defining the problem}  Objective \ref{understand-problem} requires a precise definition of the problem you are solving.  Don't forget to reference good source material \citep{henning_schulzrinne} and \citep{talbot2013}.  See section \ref{proposed}.

\paragraph{Possible Solutions} Discuss the possible solutions, compare the
alternatives, and select the one to use for the  implementation.

\subsection{List of Appendices}
What Appendices you will include.  A copy of the TOR should be the first, followed by the Ethics form and the Risk Assessment.

Others might include design documentation, code listings, tables of results (if too large to include in the main text).

\section{Marking Scheme}
The marking scheme sets out what criteria we are going to use for the project.

\paragraph{Project Type} General Computing or Software Engineering projects

\paragraph{Project Report}  State which chapters constitute the \emph{Analysis}, the \emph{Synthesis}, and the \emph{Evaluation}.  This help me when marking to know when to stop reading one section and put a mark down for it.

\paragraph{Product}  List the deliverables that make up the \emph{Product}.  Code, design, requirements specifications, test plans, etc.

For the \emph{Fitness for Purpose} and \emph{Build Quality}  list the critera used to asses the product by

\subparagraph{Fitness for Purpose}~
\begin{itemize}
	\item meet requirements identified
	\item other appropriate measures
\end{itemize}

\subparagraph{Build Quality}~
\begin{itemize}
	\item Requirements specification and analysis
	\item Design Specification
	\item Code quality
	\item Test plan and Results
\end{itemize}

\clearpage

\section{Project Plan}
\noindent
\rotatebox{90}{\input{Gantt}}
