\section{Formula Weights (V.1)} \label{Weights}

The following weights are proposed for initial testing.

\begin{table}[H]
\begin{tabular}{ll}
Response Code & Weights \\
400           & +0.5    \\
401           & +5.0    \\
403           & +2.0    \\
429           & +2.0    \\
500           & +0.2    \\
200           & -1.0   
\end{tabular}
\end{table}

If the URL contains 'Wp-Admin' then we shall apply an additional 3 to the risk related factor. If the URL contains 'login' in any case format an application of an additional 2 will be added to the risk factor. If the server responds with no data we shall apply an additional risk factor of 6.
%the casing format of these particular requests should make no difference, for example a request for a page containing LOGIN, or Login would be treat the same in terms of risk level.

%It was noted that some header codes were not included in the initial formula and hence a second draught was appropriate to include other error messages that may also be indicative of an enhanced risk factor.
\section{Formula Weights (V.2)}

After testing it was noted that some header codes were not included in the initial formula and hence a second draught was appropriate to include other error messages that may also be indicative of an enhanced risk factor.
The following weights are proposed for initial testing.

\begin{table}[H]
\begin{tabular}{ll}
Response Code & Weights \\
400           & +0.5    \\
401           & +5.0    \\
404           & +? \\
403           & +2.0    \\
429           & +2.0    \\
500           & +0.2    \\
200           & -1.0   
\end{tabular}
\end{table}

If the URL contains 'Wp-Admin' then we shall apply an additional 3 to the risk related factor. If the URL contains 'login' in any case format an application of an additional 2 will be added to the risk factor. If the server responds with no data we shall apply an additional risk factor of 6.