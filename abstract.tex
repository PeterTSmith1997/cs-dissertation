%!TeX root=Dissertation.tex

\chapter{Abstract}

This thesis identifies that there is a general lack of research into low bandwidth attacks, and there are many inherent difficulties in detecting instances of such attacks. Much of the previously available literature proposed methods for detecting low rate attacks that contained certain fundamental flaws. Many of these methods were not built with scaling in mind and would be unable to function on servers that run more than one website. The main focus of this research was to create a new approach to low rate bandwidth detection that can function at a website admin level.

This work looks at data over a long period of time and is therefore better to identify malicious traffic, specifically when looking at low rate bandwidth attacks. The methodologies that are published online have several weaknesses and therefore are not applicable. No literature could be found during this study, looking at the effects of having multiple websites on a server and the ability to detect attacks on a single website, thus, the implementation of such technologies are needed. The details pertaining to their flaws will be discussed elsewhere in this paper.

This thesis reviews a range of modern detection techniques in order to better understand potential flags for malicious low rate activity. Literature reviewing fully automated and human moderated system approaches has been reviewed in order to assess the best results in detection accuracy. A methodology for detecting low rate bandwidth attacks has been structured using a formulaic principal that reads log files and interprets the risk factor of incoming traffic. The literature review revealed a number of usability issues when creating a user interface for malicious detection, in particular when novice users are involved. In order to combat this, a simple user interface was created to work hand in hand with the software that was designed with novice users in mind. A human moderator, even with little or no Cyber security knowledge, can interpret the data from the software and root out any rare errors or false alarms generated by the formula.

The results were conclusive with a good accuracy in detecting low rate bandwidth attacks from a limited data set. The software that was created as a part of this research has little or no effect on computational resource on a target website. The testing, as part of this research, assessed a one month period of log file activity across 3 websites. The results returned a particularly low margin of error for false alerts. As mentioned, a human moderator with access to the log file data would eliminate the few false alerts generated by the system.
