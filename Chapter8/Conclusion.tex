\section{Conclusion}

This work has shown that there needs to be more research around the detection of Low-rate bandwidth attacks, the detection methods proposed in chapter 2 by Adi and Tripathi would seems to indicate that there is a problem with detecting and mitigating low-rate bandwidth attacks especially in relation to HTTP2 protocols. Adi proposed strategy of monitoring servers resource utilisation does not scale well into a shared environment. The work done by Adi shows that low rate bandwidth attacks can not be detected in real time unless the server is offline as result of the attack therefore, as proposed in this paper, by looking at the log data of a website over a longer period of time, attacks can be more easily detected and action can be taken.

This work also shows that high rate bandwidth attacks can be easily mitigated and should not pose a problem to website owners if their security systems are implemented properly. This work also shows that due to the scale of high rate bandwidth attacks these are better mitigated using a network of proxy servers. 

This work has also demonstrated that a lack of good quality research has been done in looking at whether or not users can actually identify an attack without much training. This paper has also shown that using a combination of computers and humans, we can strike a balance between human being informed and taking decision while letting the computer do the mathematical calculation to define the risk. It is also suggested by the author of the paper that the human needs to be kept in the loop to give the system context and to keep them informed about their web traffic. 

%Due to the fact that fake google bots or fake search engine bots exist online, this first check could be altered in the future to look at whether a bot instance is legitimate. The decision was made not to implement this as part of the formula during this research, we shall discuss the reason for this decision during the final conclusions of the research.

%It is worth mentioning that due to the increasing use of VPN activity online, it may be a good idea to add an extention to this project. This would entail an extention of the database to include VPN IP addessses much like the recording of known bots. However, people can still flag IP's as a bot. This cannot be categorised as a VPN.

%A separate risk factor needs to be worked in for VPN's.