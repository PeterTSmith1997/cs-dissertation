\section{Evaluation of Process}

Due to factors beyond my control I was unable to start my project for the first six weeks of term. This initially had a significant impact on my schedule and flowchart. These documents are outlined within the terms of reference. This delayed the writing of the analysis and literature review, however, after consultation with my supervisor, it was decided that it would be appropriate to begin writing some of the software code for the potential project during this period of absence from university. Working from home during this period was instrumental in keeping the project moving. While tweaks and changes were required after the literature review, these alterations were easy to implement within the initial draught code. After this period of absence the project was well managed and I was able to get back on track with relative ease. At the beginning of the second semester I was able to pass my scheduled expectations and pushed on towards editing.

As can be seen in (appendix J) I kept continuous logs of any meetings held with my supervisor during the term of the project. I found these meetings extremely helpful, the feedback during the meetings was instrumental in keeping myself ahead of schedule towards the latter part of the dissertation. In general I was pleased with my timekeeping and scheduling during the project. I learned that I was efficient at managing my time to keep in line with soft deadlines. My allocations for time spent between coding and report writing were well defined and helped to keep both aspects of the research up to date. During the research I found an extraordinary wealth of literature that delves into the world of human/computer interaction. In particular the work carried out by researchers regarding user expertise and related performance with IT systems. I found this particular topic interesting and have considered future research into the field.

Overall, I believe the majority of objectives for the project were achieved. After reviewing the terms of reference the following adaptations were made whilst undertaking the project. After careful consideration the ERD and class diagram were considered inappropriate for the software; This was mainly due to the fact that the software and code was easier to formulate than had initially been imagined.

\subsection{Impacts of Covid-19}

Towards the back end of the project there was a viral outbreak that elevated to pandemic status around the globe; this had an adverse effect on university facilities, and eventually the campus was forced to close. On the 12/03/2020 I was regrettably unable to attend Northumbria university and began remote working, this had an impact on the back end activities planned for this project. Because of this I had to email my dissertation supervisor and rearrange plans for our weekly consultations. By this stage I had the majority of the written and practical work done, however, the code remained without comment and the software required for altering this was based on campus computers; this presented an unforeseen barrier that required additional contingency planning to overcome. I communicated with both the management at my care provider and my dissertation supervisor in order to make relevant changes. I managed to source and install eclipse on my home computer in order to continue with the code commenting process. Although the outbreak of the virus was terribly timed, It presented me with a situation which tested my ability to problem solve in a crisis situation. 


Another limitation that arose due to the viral outbreak was that the ability to conduct research around user testing. User testing was planned to take place from the project's beginning however, after reviewing the work by Ben Asher (\citeyear{ben2015effects}), this was more complex than I had initially planned. The testing may have been possible utilizing a combination of final year and first year students in order to get the feedback of experts and non-experts. Due to the current Covid-19 pandemic, all face-to-face interactions have been suspended, although the testing could have been done via the internet; this would have made an already complex testing methodology even more difficult. I have considered completing a full user interface testing for the software as a continuation project. The future work would entail the formulation of two user groups similar to the work carried out by \citeauthor{ben2015effects}. The user groups would be defined by their knowledge of cyber security. It is believed that this would be an ideal way to assess whether a UI was in a fit state for potential industrial release. 






 

