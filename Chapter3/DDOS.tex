%!TeX root=Dissertation.tex

\section{DDOS attacks} \label{attack2}
%https://reader.elsevier.com/reader/sd/pii/S0140366417303791?token=098B9C0758047352DF30300332E273B21DB9C488D1FE2D089522058BA6FD67DC86B8FC286ABBA0C30D94E897E4808B93

High-bandwidth attacks can have serious implications for businesses as they can take a website offline, therefor, the early detection of high-bandwidth DDoS attacks is crucial. An example of prompt mitigation of a bandwidth DDoS attack can be seen in the Githhub Case 2018; where by a 1.3 Tps high-bandwidth attack was launched against Githhub bringing services to a stop for 9 minutes. \cite{Githhubattacks} This case shows how important it is to have prompt detection and mitigation of high rate DDoS attacks. Due to the commercial nature of providing mitigation technologies for high-bandwidth attacks, there is very little publicly available material on the architecture and design of defence mechanisms. However, some rudimentary detection and mitigation techniques have been proposed publicly by academics. 

In the modern world businesses will take the networking approach of either using a reverse proxy via a provider such as cloudflare, or by the use of in house networking mechanics such as SDN or NFV. The majority of the papers found while researching High bandwidth attacks are looking into detection methodology while using SDN networking. SDN networking is more important now due to the adaptation of the cloud by using SDN networking (FIND LOST PAPER --- LINK ABOVE). The cloud allows the business to automatically scale the resources on a server automatically to deal with peaks in traffic. However, if an attack is not managed properly the server will continue to scale therefor increasing cost for the organisation. 

Haopei Wang notes in 2015, when the emergence of SDN networking was relatively new, that SDN abstracts the physical layer from the hardware layer therefor the control of the network is done at the software. This therefor leads to a more fine grained control of the network and opens up the opportunity for new defence mechanisms. The team developed a DDoS identifying and mitigation process that they named FLOODGUARD. The process involves a two stage saturation mitigation technique involving a Proactive flow rule analyzer and a packet migration module. Through a comprehensive detection algorithm at the migration agent, looking at Memory depletion and rate of packet\_in messages, malign traffic is separated from genuine traffic and mitigated using handling dynamics. (check d. handling dynamics)


The most recent paper available was written by \citeauthor{ahalawat2019entropy} He assessed the use of SDN networking and theorised that entropy could be used as a measurement of randomness in an effort to detect DDoS traffic. A mitigation technique was introduced by limiting the rate of packet flow allowed to flow to the switch, while in turn, Attack control plane bandwidth is prevented by limiting the inflow rate to the controller. The team concluded that SDN networks seemed to be in particular most vulnerable to DDoS attacks that flooded the network with UDP packets. The researchers noticed that this mitigation technique had the ability to be implemented much earlier than some other techniques that had been previously researched and developed. It was also successful in restoring the use of services to 'normal traffic.' \cite{ahalawat2019entropy}

The software for this dissertation will not have a primary focus on focusing on high rate bandwidth attacks, due to the fact that they require immediate mitigation to avoid disruption to web services. Also as discussed earlier in this section, detection and mitigation tend to be handled at a control plane phase through networking, or at the filter stage if the website has an independent CDN. 